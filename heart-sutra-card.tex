\documentclass{chant-card}

\usepackage{myrule}

\setlength{\parskip}{12pt plus 2pt}

\begin{document}
\fontsize{15pt}{20pt}\selectfont

\section{Heart of Great Perfect Wisdom Sutra}

Avalokiteshvara Bodhisattva, when deeply practicing prajna paramita, clearly
saw that all five aggregates are empty and thus relieved all suffering.
Shariputra, form does not differ from emptiness, emptiness does not differ from
form. Form itself is emptiness, emptiness itself form. Sensations, perceptions,
formations, and consciousness are also like this. Shariputra, all dharmas are
marked by emptiness; they neither arise nor cease, are neither defiled nor pure,
neither increase nor decrease. Therefore, given emptiness, there is no form, no
sensation, no perception, no formation, no consciousness; no eyes, no ears, no
nose, no tongue, no body, no mind; no sight, no sound, no smell, no taste, no
touch, no object of mind; no realm of sight\dots\ no realm of mind
consciousness. There is neither ignorance nor extinction of ignorance\dots\
neither old age and death, nor extinction of old age and death; no suffering, no
cause, no cessation, no path; no knowledge and no attainment.  With nothing to
attain, a bodhisattva relies on prajna paramita and thus the mind is without
hindrance. Without hindrance, there is no fear.  Far beyond all inverted views,
one realizes nirvana. All buddhas of past, present, and future rely on prajna
paramita and thereby attain unsurpassed, complete, perfect enlightenment.
Therefore know the prajna paramita as the great miraculous mantra, the great
bright mantra, the supreme mantra, the incomparable mantra, which removes all
suffering and is true, not false.  Therefore we proclaim the prajna paramita
mantra, the mantra that says Gate Gate Paragate Parasamgate Bodhi Svaha.

\begin{center}
  All buddhas, ten directions, three times\\
  All honored ones, bodhisattvas, mahasatvas\\
  Wisdom beyond wisdom, maha prajna paramita
\end{center}

\end{document}
