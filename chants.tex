
\newcommand{\paliRefuges}{%
Buddham saranam gacchami\\
Dhammam saranam gacchami\\
Sangham saranam gacchami

Dutiyampi buddham saranam gacchami\\
Dutiyampi dhammam saranam gacchami\\
Dutiyampi sangham saranam gacchami

Tatiyampi buddham saranam gacchami\\
Tatiyampi dhammam saranam gacchami\\
Tatiyampi sangham saranam gacchami
}

\newcommand{\allBuddhas}{%
\begin{center}
All Buddhas, Ten Directions, Three Times \bigspace\largebell\\
All Honored Ones, Bodhisattvas, Mahasattvas \bigspace\largebell\\
Wisdom Beyond Wisdom, Maha Prajna Paramita \bigspace\bok
\end{center}
}

\newcommand{\songOfTheJewelMirrorSamadhi}{%
\begin{chant}{Song of the Jewel Mirror Samadhi}

The teaching of thusness has been intimately communicated by Buddhas and
ancestors. Now you have it, so keep it well. Filling a silver bowl with snow,
hiding a heron in the moonlight, \largebell\ taken as similar they're not the
same; when you mix them, you know where they are. The meaning is not in the
words, yet it responds to the inquiring impulse. Move and you are trapped; miss
and you fall into doubt and vacillation. Turning away and touching are both
wrong, for it is like a massive fire. Just to depict it in literary form is to
stain it with defilement. It is bright just at midnight, it doesn't appear at
dawn. It acts as a guide for beings, its use removes all pains. Although it is
not fabricated, it is not without speech. It is like facing a jewel mirror;
form and image behold each other---you are not it, in truth it is you. Like a
babe in the world, in five aspects complete; it does not go or come, nor rise
nor stand. ``Baba wawa''---is there anything said or not? Ultimately it does
not apprehend anything because its speech is not yet correct. It is like the
six lines of the illumination hexagram: relative and ultimate interact---piled
up, they make three, the complete transformation makes five. It is like the
taste of the five-flavored herb, like a diamond thunderbolt. Subtly included
within the true, inquiry and response come up together. Communing with the
source, travel the pathways, embrace the territory and treasure the road.
Respecting this is fortunate; do not neglect it. Naturally real yet
inconceivable, it is not within the province of delusion or enlightenment. With
causal conditions, time and season, quiescently it shines bright. In its
fineness it fits into spacelessness, in its greatness it is utterly beyond
location. A hairsbreadth's deviation will fail to accord with the proper
attunement. Now there are sudden and gradual in which teachings and approaches
arise. Once basic approaches are distinguished, then there are guiding rules.
But even though the basis is reached and the approach comprehended, true
eternity still flows. Outwardly still while inwardly moving, like a tethered
colt, a trapped rat---the ancient sages pitied them and bestowed upon them the
teaching. According to their delusions, they called black as white; when
erroneous imaginations cease, the acquiescent mind realizes itself. If you want
to conform to the ancient way, please observe the sages of former times. When
about to fulfill the way of Buddhahood, one gazed at a tree for ten eons. Like
a battle-scarred tiger, like a horse with shanks gone gray. Because there is
the common, there are jewel pedestals, fine clothing; because there is the
startlingly different, there are house cat and cow. Yi with his archer's skill
could hit a target at a hundred paces. But when arrow-points meet head on, what
has this to do with the power of skill? When the wooden man begins to sing, the
stone woman gets up dancing; it's not within reach of feeling or
discrimination---how could it admit of consideration in thought? Ministers
serve their lords, children obey their parents; not obeying is not filial and
not serving is no help. Practice secretly \smallbell\ working within, like a
fool, like an idiot \smallbell\ just to continue in this way is called the host
within the host.  \bok
\end{chant}}

\newcommand{\ancestorsShort}{%
\begin{chant*}
Bibashi Butsu Daiosho \clank\ Shiki Butsu Daiosho \clank\ Bishafu Butsu Daiosho
\clank\ Kuruson Butsu Daiosho \clank\ Kunagonmuni Butsu Daiosho \clank\ Kasho
Butsu Daiosho \clank\ Shakamuni Butsu Daiosho Makakasho Daiosho Ananda Daiosho
Shonawashu Daiosho Ubakikuta Daiosho Daitaka Daiosho Mishaka Daiosho Vashumitsu
Daiosho Butsudanandai Daiosho Fudamitta Daiosho Barishiba Daiosho Funayasha
Daiosho Anabotei Daiosho Kabimare Daiosho Nagyaharajuna Daiosho Kanadaiba
Daiosho Ragorata Daiosho Sogyanandai Daiosho Kayashata Daiosho Kumorata Daiosho
Shayata Daiosho Vashubanzu Daiosho Manura Daiosho Kakurokuna Daiosho
Shishibodai Daiosho Bashashita Daiosho Funyomitta Daiosho Hannyatara Daiosho
Bodaidaruma Daiosho Taiso Eka Daiosho Kanchi Sosan Daiosho Dai-i Doshin Daiosho
Daiman Konin Daiosho Daikan Eno Daiosho Seigen Gyoshi Daiosho Sekito Kisen
Daiosho Yakusan Igen Daiosho Ungan Donjo Daiosho Tozan Ryokai Daiosho Ungo Doyo
Daiosho Doan Dohi Daiosho Doan Kanchi Daiosho Ryozan Enkan Daiosho Taiyo Kyogen
Daiosho Toshi Gisei Daiosho Fuyo Dokai Daiosho Tanka Shijun Daiosho Choro
Seiryo Daiosho Tendo Sokaku Daiosho Seccho Chikan Daiosho Tendo Nyojo Daiosho
Eihei Dogen Daiosho Koun Ejo Daiosho Tettsu Gikai Daiosho Keizan \clank\ Jokin
\clank\ Daiosho \bok
\end{chant*}
}

\newcommand{\femaleAncestors}{%
\begin{chant*}
Acharya Mahapajapati \clank\  Acharya Mitta \clank\  Acharya Yasodhara \clank\ 
Acharya Tissa Acharya Sujata Acharya Sundari-nanda Acharya Vaddhesi Acharya
Patachara Acharya Visakha Acharya Singalaka-mata Acharya Khema Acharya
Uppalavanna Acharya Samavati Acharya Uttara Acharya Chanda Acharya Uttama
Acharya Bhadda Kundalakesa Acharya Nanduttara Acharya Dantika Acharya Sakula
Acharya Siha Acharya Dhammadinna Acharya Kisagotami Acharya Ubbiri Acharya
Isidasi Acharya Bhadda Kapilani Acharya Mutta Acharya Sumana Acharya Dhamma
Acharya Chitta Acharya Anopama Acharya Sukka Acharya Sama Acharya Utpalavarna
Acharya Shrimala Devi Acharya Congchi Acharya Lingzhao Acharya Moshan Liaoran
Acharya Liu Tiemo Acharya Miaoxin Acharya Daoshen Acharya Shiji Acharya Zhi'an
Acharya Huiguang Acharya Kongshi Daoren Acharya Yu Daopo Acharya Huiwen Acharya
Fadeng Acharya Wenzhao Acharya Miaodao Acharya Zhitong Acharya Zenshin Acharya
Zenzo Acharya Ezen Acharya Ryonen Acharya Egi Acharya Shogaku Acharya Ekan
Acharya Shozen Acharya Mokufu Sonin Acharya Myosho Enkan Acharya Ekyu Acharya
Eshun Acharya Soshin Acharya \clank\  Soitsu Acharya \clank\  Chiyono \bok
\end{chant*}
}

\newcommand{\jiHoSan}{%
\begin{center}
Ji ho san shi i shi fu \bigspace\largebell\\
shi son bu sa mo ko sa \bigspace\largebell\\
mo ko ho ja ho ro mi \bigspace\null\bigspace\null
\end{center}

\smallBellRolldown
}

\newcommand{\makaHannyaHaramittaShingyo}{%
\begin{chant}{Maka Hannya Haramitta Shingyo}

Kan ji zai bo satsu gyo jin han-nya ha ra mi ta ji sho ken go on kai ku do
is-sai ku yaku sha ri shi shiki fu i ku ku fu i shiki shiki soku ze \largebell\ 
ku ku soku ze shiki ju so gyo shiki yaku bu nyo ze sha ri shi ze sho ho ku so
fu sho fu metsu fu ku fu jo fu zo fu gen ze ko ku chu mu shiki mu ju so gyo
shiki mu gen ni bi zes-shin ni mu shiki sho ko mi soku ho mu gen kai nai shi mu
i shiki kai mu mu myo yaku mu mu myo jin nai shi mu ro shi yaku mu ro shi jin
mu ku shu metsu do mu chi yaku mu toku i mu sho tok-ko bo dai sat-ta e han nya
ha ra mit ta ko shin mu kei ge mu kei ge ko mu u ku fu on ri is-sai ten do mu
so ku gyo ne han san ze sho butsu e han-nya ha ra mit ta ko toku a noku ta ra
sam myaku sam bo dai ko chi han-nya ha ra mi ta ze dai jin shu ze dai myo shu
ze mu jo shu ze mu to do shu no jo is-sai ku shin jitsu fu ko ko setsu han-nya
ha ra mit ta shu soku setsu shu watsu \smallbell\ gya tei gya tei ha ra gya tei
\smallbell\ hara so gya tei bo ji sowa ka han-nya shin gyo \bok
\end{chant}
}
\newcommand{\shosaimyoKichijoDharani}{%
\begin{chant}{Shosaimyo Kichijo Dharani}

No mo san man da moto nan oha ra chi koto sha sono nan to ji to \largebell\ en
gya gya gya ki gya ki un nun shifu ra shifu ra hara shifu ra hara shifu ra
chishu sa chishu sa chishu ri chishu ri sowa ja sowa ja sen chi gya shiri ei
somo-ko

No mo san man da moto nan oha ra chi koto sha sono nan to ji to en gya gya gya
ki gya ki un nun shifu ra shifu ra hara shifu ra hara shifu ra chishu sa chishu
sa chishu ri chishu ri sowa ja sowa ja sen chi gya shiri ei somo-ko

No mo san man da moto nan oha ra chi koto sha sono nan to ji to en gya gya gya
ki gya ki un nun shifu ra shifu ra hara shifu ra hara shifu ra chishu sa chishu
sa \smallbell\ chishu ri chishu ri sowa ja sowa ja \smallbell\ sen chi gya
shiri ei so moko \bok
\end{chant}
}

\newcommand{\weDedicateThisMeritTo}[1]{%
Looking upward we deeply wish for Buddha's true compassion. Bowing down
we ask the illumination of Buddha's understanding.

Chanting the #1.

We dedicate this merit to:

\begin{outdent}
\smallbell$\uparrow$

Our original ancestor in India, great teacher Shakyamuni Buddha,\\
Our first female ancestor, great teacher Mahapajapati,\\
The first ancestor in China, great teacher Bodhidharma,\\
The first ancestor in Japan, great teacher Eihei Dogen,\\
Our compassionate founder, great teacher Shogaku Shunryu.

$\downarrow$\smallbell

\end{outdent}

Gratefully we offer this virtue to all beings $\sim$ \largebell
}
\newcommand{\harmonyOfDifferenceAndEquality}{%
\begin{chant}{Harmony of Difference and Equality}

The mind of the great sage of India is intimately transmitted from west to
east. While human faculties are sharp or dull, the way has no northern or
southern ancestors. \largebell\ The spiritual source shines clear in the light,
the branching streams flow on in the dark. Grasping at things is surely
delusion, according with sameness is still not enlightenment. All the objects
of the senses interact and yet do not. Interacting brings involvement,
otherwise each keeps its place. Sights vary in quality and form, sounds differ
in pleasing or harsh. Refined and common speech come together in the dark,
clear and murky phrases are distinguished in the light. The four elements
return to their natures just as a child turns to its mother.
\doshidependent{Fire heats, wind moves, water wets, earth is solid, eye and
sights, ear and sounds, nose and smells, tongue and tastes. Thus with each and
every thing, depending on these roots the leaves spread forth. Trunk and
branches share the essence, revered and common each has its speech. In the
light there is darkness, but don't take it as darkness. In dark there is light,
but don't see it as light. Light and dark oppose one another like the front and
back foot in walking. Each of the myriad things has its merit expressed
according to function and place.} Phenomena exist, box and lid fit, principle
responds, arrow points meet.  Hearing the words understand the meaning; don't
set up standards of your own.  If you don't understand the way right before
you, how will you know the path as you walk? Progress is not a matter of far or
near, but if you are confused \smallbell\ mountains and rivers block your way.
\smallbell\ I respectfully urge you who study the mystery; do not pass your
days and nights in vain. \bok
\end{chant}
}

\newcommand{\ancestorsLong}{%
\begin{chant*}
Bibashi Butsu Daiosho \clank\ Shiki Butsu Daiosho \clank\ Bishafu Butsu Daiosho
\clank\ Kuruson Butsu Daiosho \clank\ Kunagonmuni Butsu Daiosho \clank\ Kasho
Butsu Daiosho \clank\ Shakamuni Butsu Daiosho Makakasho Daiosho Ananda Daiosho
Shonawashu Daiosho Ubakikuta Daiosho Daitaka Daiosho Mishaka Daiosho Vashumitsu
Daiosho Butsudanandai Daiosho Fudamitta Daiosho Barishiba Daiosho Funayasha
Daiosho Anabotei Daiosho Kabimare Daiosho Nagyaharajuna Daiosho Kanadaiba
Daiosho Ragorata Daiosho Sogyanandai Daiosho Kayashata Daiosho Kumorata Daiosho
Shayata Daiosho Vashubanzu Daiosho Manura Daiosho Kakurokuna Daiosho
Shishibodai Daiosho Bashashita Daiosho Funyomitta Daiosho Hannyatara Daiosho
Bodaidaruma Daiosho Taiso Eka Daiosho Kanchi Sosan Daiosho Dai-i Doshin Daiosho
Daiman Konin Daiosho Daikan Eno Dai\-osho Seigen Gyoshi Daiosho Sekito Kisen
Daiosho Yakusan Igen Daiosho Ungan Donjo Daiosho Tozan Ryokai Daiosho Ungo Doyo
Daiosho Doan Dohi Daiosho Doan Kanchi Daiosho Ryozan Enkan Daiosho Taiyo Kyogen
Daiosho Toshi Gisei Daiosho Fuyo Dokai Daiosho Tanka Shijun Daiosho Choro
Seiryo Daiosho Tendo Sokaku Daiosho Seccho Chikan Daiosho Tendo Nyojo Daiosho
Eihei Dogen Daiosho Koun Ejo Daiosho Tettsu Gikai Daiosho Keizan Jokin Daiosho

Gasan Joseki Daiosho Taigen Soshin Daiosho Baizan Mompon Daiosho Jochu Tengin
Daiosho Shingan Doku Daiosho Senso Esai Daiosho Iyoku Choyu Daiosho Mugai
Keigon Daiosho Nenshitsu Yokaku Daiosho Sesso Hoseki Daiosho Taiei Zesho
Daiosho Nampo Gentaku Daiosho Zoden Yoko Daiosho Tenyu Soen Daiosho Ken'an
Junsa Daiosho Chokoku Koen Daiosho Senshu Donko Daiosho Fuden Gentotsu Daiosho
Daishun Kan'\-yu Daiosho Tenrin Kanshu Daiosho Sessan Tetsuzen Daiosho Fuzan
Shunki Daiosho Jissan Mokuin Daiosho Sengan Bonryu Daiosho Daiki Kyokan Daiosho
Enjo Gikan Daiosho Shoun Hozui Daiosho Shizan Tokuchu Daiosho Nanso Shinshu
Daiosho Kankai Tokuon Daiosho Kosen Baido Daiosho Gyakushitsu Sojun Daiosho
Butsumon Sokaku Daiosho Gyokujun So-on \clank\ Daiosho Shogaku \clank\ Shunryu
Daiosho \bok
\end{chant*}
}

\newcommand{\enmeiJukkuKannonGyo}{%
\begin{chant}{Enmei Jukku Kannon Gyo}

(1x) Kan ze on na mu butsu yo butsu u in yo butsu u en \largebell\ bup-po so en
jo raku ga jo cho nen kan ze on bo nen kan ze on nen nen ju shin ki nen nen fu
ri shin

(2 \& 3x) Kan ze on na mu butsu yo butsu u in yo butsu u en bup-po so en jo
raku ga jo cho nen kan ze on bo nen kan ze on nen nen ju shin ki nen nen fu ri
shin

(4 \& 5x) Kan ze on na mu butsu yo butsu u in yo butsu u en bup-po so en jo
raku ga jo cho nen kan ze on bo nen kan ze on nen nen ju shin ki nen nen fu ri
shin \largebell

(6x) Kan ze on na mu butsu yo butsu u in yo butsu u en bup-po so en jo raku ga
jo cho nen kan ze on bo nen kan ze on nen nen ju shin ki nen nen fu ri shin

(7x) Kan ze on na mu butsu yo butsu u in yo butsu u en bup-po so en jo raku ga
jo \smallbell\ cho nen kan ze on bo nen kan ze on nen nen ju shin ki \smallbell\ 
nen nen fu ri shin \bok
\end{chant}
}

\newcommand{\lovingKindnessMeditation}{%
\begin{chant}{Loving Kindness Meditation}

This is what should be accomplished by the one who is wise, who seeks the good,
and has obtained peace. \largebell\ Let one be strenuous, upright, and sincere,
without pride, easily contented, and joyous. Let one not be submerged by the
things of the world. Let one not take upon oneself the burden of riches. Let
one's senses be controlled. Let one be wise but not puffed up and let one not
desire great possessions even for one's family. Let one do nothing that is mean
or that the wise would reprove. May all beings be happy. May they be joyous and
live in safety. All living beings, whether weak or strong, in high or middle or
low realms of existence, small or great, visible or invisible, near or far,
born or to be born, may all beings be happy. Let no one deceive another nor
despise any being in any state. Let none by anger or hatred wish harm to
another. Even as a mother at the risk of her life watches over and protects her
only child, so with a boundless mind should one cherish all living things,
suffusing love over the entire world, above, below, and all around, without
limit. So let one cultivate an infinite good will toward the whole world.
Standing or walking, sitting or lying down, during all one's waking hours, let
one practice the way with gratitude. Not holding to fixed views, endowed with
insight, freed from sense appetites, one who achieves \smallbell\ the way will
be freed \smallbell\ from the duality of birth and death. \bok
\end{chant}
}

\newcommand{\heartOfGreatPerfectWisdomSutra}{%
\begin{chant}{Heart of Great Perfect Wisdom Sutra}

Avalokiteshvara Bodhisattva, when deeply practicing prajna paramita, clearly
saw that all five aggregates are empty and thus relieved all suffering.
Shariputra, form does not differ from emptiness, emptiness does not differ from
form. \largebell\ Form itself is emptiness, emptiness itself form. Sensations,
perceptions, formations, and consciousness are also like this. Shariputra, all
dharmas are marked by emptiness; they neither arise nor cease, are neither
defiled nor pure, neither increase nor decrease. Therefore, given emptiness,
there is no form, no sensation, no perception, no formation, no consciousness;
no eyes, no ears, no nose, no tongue, no body, no mind; no sight, no sound, no
smell, no taste, no touch, no object of mind; no realm of sight\dots\ no realm
of mind consciousness. There is neither ignorance nor extinction of
ignorance\dots\ neither old age and death, nor extinction of old age and
death; no suffering, no cause, no cessation, no path; no knowledge and no
attainment.  \doshidependent{With nothing to attain, a bodhisattva relies on
prajna paramita and thus the mind is without hindrance. Without hindrance,
there is no fear.  Far beyond all inverted views, one realizes nirvana. All
buddhas of past, present, and future rely on prajna paramita and thereby attain
unsurpassed, complete, perfect enlightenment.} Therefore know the prajna
paramita as the great miraculous mantra, the great bright mantra, the supreme
mantra, the incomparable mantra, which removes all suffering and is true, not
false.  Therefore we proclaim the prajna paramita mantra, the mantra that says
\smallbell\ Gate Gate Paragate \smallbell\ Parasamgate Bodhi Svaha \bok
\end{chant}
}

\newenvironment{service}{%
\doshi Perform a standing bow behind the bowing mat, approach the altar and
offer one long stick of incense and two pinches of chip incense.

Step back and bows to altar, then walk back around to the bowing mat and put
out bowing cloth.

\doan Ring the small bell as the doshi walks around the mat as follows:
\doshiBowingClothRolldown
The ringdown ends as the doshi drops the corners of the bowing cloth.
\doshi Perform nine full bows.
\doan Ring the large bell once for each bow. The first eight rings are
followed by muffles as the doshi's hands touch down, the ninth is followed by a
second ring as the doshi's forehead touches down.
\firstBows
\sangha Bow with the Doshi.
\doshi Stand up, leaving bowing cloth on cushion.
\sangha \paliRefuges
\doshi Step back, do a standing bow.
\sangha Switch to shashu.
\doshi Approach the altar and offer more loose incense. Bow to the altar, then return to the bowing mat.
\doan Ring the large bell once as the doshi bows to the altar, then ring the
small bell twice as the doshi passes the cushion.
\takeOutChantBookBells
\sangha Sit on your cushion.
\doshi Perform three full bows.
\doan Ring the large bell for the first two bows, and bok the bell on the
third.
\secondBows
}{%
\par
\jiHoSan

\doan End the ringdown when everyone is ready to bow.
\doshi Lead the sangha in three bows.
\doan Ring the small bell for each of the three bows, with a fourth ring as the
doshi's forehead touches the cushion on the third bow.
\lastBows

\doshi Pick up bowing cloth, wait for the sangha to put away their cushions.

When the sangha is ready, bow behind your cushion.
\doan Ring the small bell. \smallbell
\tenken Open the zendo doors.
\doshi Bow again at the zendo threshold.
\doan Ring the small bell. \smallbell
\doshi Step back, bows again.
\doan Ring the small bell twice, about two seconds apart.
\bline{\hfill\smallbell\hfill\smallbell\hfill\null}
\doan Lead the sangha in a shashu bow.

Lead the sangha out of the zendo.
}
