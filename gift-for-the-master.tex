\documentclass{book}
\usepackage{bookdesign}
\usepackage{kpfonts}
\usepackage[paperwidth=5in,paperheight=7in,top=1in,bottom=1in,outer=1in,inner=.75in,includefoot]{geometry}
\usepackage{parskip}

\renewcommand{\chapter}[1]{
  \cleardoublepage
  \vspace*{1in}
  {\par\noindent\raggedright \Large #1\par}
  \vspace*{.5in}
}

\begin{document}
\pagestyle{empty}
\frontmatter
\begin{titlepage}
  \Large
  A Gift for the Master

  \bigskip

  Bonnie Sarmiento
\end{titlepage}

\begin{colophon}
  \copyright\ 2015 by Bonnie Sarmiento\\
  1972 Rock St, Mountain View, CA 94043

  Sarmiento, Bonnie\\
  Zen, Buddhism

  All rights reserved. No part of this book may be\\
  reproduced or transmitted, unless done with good\\
  humor and a smile.

  Printed and bound by Scott Williams.
\end{colophon}

\begin{dedication}
  \null\vfill\em
  For Roshi Les Kaye,\\
  on your 50th Zeniversary
  \vfill\null
\end{dedication}

\mainmatter
\pagestyle{plain}

\chapter{A Gift for the Master}

Once upon a time, in Valley Siliconia, there was an ancient temple by the name
of `Kannon Do.' `Kannon Do' meant `Place of Compassion,' and many people
traveled to this temple to hear words of wisdom from the abbot. The abbot of
the temple was a wise, old zen master, and word got around that soon it would
be his 50th anniversary of practicing The Way.

The priests and students of the temple at once set about planning a great
celebration. ``We will invite everyone!'' they said. ``We will shower him with
gifts!'' ``He will be so surprised!'' But while they went about eagerly with
preparations, one young monk sat troubled in the corner.

\pagebreak

``A gift for the master?'' she thought, ``What could I possibly give to the
master that would be enough to thank him for 50 years of dedicated practice?''
She faced the wall, and pondered this some more. ``He doesn't need any new
clothes.  His robes are always well cared for. He already has scrolls and
paintings. He already has ceremonial cups and incense. He has a home, a garden,
and a gym membership at 24 Hour Fitness!'' And the more she thought about it,
the more confounded she became, until it was a koan that troubled her day and
night. 

``What gift can you give to a zen master, who already has everything?''

She contemplated this koan for days. Days turned into weeks. But every idea
that came seemed far too inadequate, too trivial to thank someone for 50 years
of their life. She was lost deep in concentration when suddenly she looked at
the calendar and realized the date of the celebration had snuck up on her. The
ceremony was tonight and she had no choice but to go to the temple
empty-handed.

People had traveled from many distant lands to come see the master that
evening. The zendo was packed with the happy faces of students and friends.
With a beautiful ceremony, the sangha thanked the master for his profound
dedication. Then, one by one, the sangha members presented their gifts of
appreciation. 

``Dear master! We know how much you take interest in the indigenous peoples.
Here, we would like to present you with the full BBC series on the
Aborigenes!'' the priestess smiled as she handed over the eloquently wrapped
documentary dvds. 

``Kind teacher!'' spoke up another young woman. ``I remember how much you
appreciate fine pottery. I have such fond memories of listening to your
teachings while we spun ceramic cups at Tassajara. Here, I have brought you
this fine, porcelain tea set from my travels in Taiwan. Please accept it as a
small token of my gratitude!''

``Beloved abbot!'' came the chorus of a group of young monks, ``To thank you,
we have brought you a most delicious triple layered cake from the finest bakery
in Valley Siliconia, with your name painted in purple frosting!'' And they
presented a glowing frosted cake, shimmering beneath golden candles.

``Dear teacher! Here, we have brought you a bouquet of fine flowers for you to
share with your wife, Miss Mary!''

``Mon am\'i! C'est un certificat pour le meilleur d\^iner au restaurant Chez
TJ's!  S'il vous pla\^it profiter de la meilleure cuisine fran\c caise de moi,
ton serviteur et ami.''

Dozens of people stepped forward presenting cards and gifts and words of
appreciation. At last, there was no one left but her, so the young monk stepped
trembling forward. She looked into the honest eyes of the old master, as he sat
cross-legged under a simple brown sash. She reached out her hands, but there
was nothing to give. 

``My hands are empty master.'' She looked down at the floor, tears brimming in
her eyes. ``I am sorry.'' 

The master looked at the young monk curiously. After a pause, he asked, ``These
empty hands, tell me, what are they good for?''

She thought a moment. ``Well, they are good for moving things. Good for washing
dishes and scrubbing floors and windows. They are good for pouring water, tea
and coffee! They are good with a pen, with a sewing needle, with a stirring
spoon! They are good with many things!'' Her eyes brightened, ``Is there
something I can do for you, master?''

He answered, ``The seeds that I planted here, will you water them?''

``Yes, master, I will water the seeds that you planted!''

``The altar I build, will you dust it for me?''

``Yes, master, I will dust the altar you built.''

``The doors to our zendo---will you open them for all seekers of The Way?''

``Yes, master, I will open the doors for all seekers of The Way!''

``And when the morning bell rings, will you just sit still?''

``Yes, master! When the morning bell rings, I will sit very still.''

The old master smiled deeply. ``These hands are a wonderful gift! I am glad you
thought of it!''

That evening the courtyard was lit with sparkling lights and the joyous voices
of celebration. And there was much cake eating and tea sipping at the sangha
house of Kannon Do.

The End.

\chapter{Acknowledgements}
I would like to thank Scott Williams for your help with formatting, printing,
and binding this booklet, to Carolina Sturgeon for all your effort planning
Les' 50 year celebration, and to the Kannon Do sangha, for your continuous
practice.

\end{document}
